%% LyX 2.1.4 created this file.  For more info, see http://www.lyx.org/.
%% Do not edit unless you really know what you are doing.
\documentclass[english]{article}
\usepackage[T1]{fontenc}
\usepackage[latin9]{inputenc}
\usepackage{babel}
\begin{document}

\section{Camins i algorismes}

Sovint, quan utilitzem un graf per modelitzar quelcom, ens interessa
poder-hi fer algunes operacions. Podem, per exemple, voler trobar
un cam� entre dos punts, rec�rrer el graf sencer o trobar el cam�
m�s curt per anar d'un v�rtex a un altre. Per aquest motiu utilitzem
els camins, que trobarem o generarem mitjn�ant diversos algorismes.
En aquesta secci� mostrar� diverses maneres de rec�rrer un graf, torbant
la manera m�s eficient per a cada cas.


\subsection{Camins}

Un cam� $p$ �s una seq��ncia finita i ordenada d'arestes que connecta
una seq��ncia ordenada de v�rtexs. Un cam� $p$ de longitud $k$ (expressat com a $l(p)=k$) entre el v�rtex inicial $v_{0}$ i el v�rtex final $v_{k}$ sempre que $v_{0}\neq v_{k}$) �s una successi� de $k$ arestes i $k+1$ v�rtexs de la forma $\overline{v_{0},v_{1}}, \overline{v_{1},v_{2}},\cdots, \overline{v_{k-1},v_{k}}$  . Per definici�, tamb� es pot representar un cam� $p$ entre $v_{0}$ i $v_{k}$ com a successi� de v�rtex $p=v_{0}v_{1}\cdots v_{k}$. En auqest cas,pot ser tractat com un graf elemental $P_{n}$. Un cas especial �s quan el cam� comenca i acaba al mateix v�rtex ($v_{0} = v_{k}$). Llavors el cam� �s un cicle, i �s l'equivalent a un graf cicle $C_{n}$. 
Quan un cam� t� totes les arestes diferents, s'anomena simple, i si a m�s t� tots els v�rtexs diferents, s'anomena elemental.

\subsection{Grafs ponderats i dirigits}


\paragraph{Grafs ponderats}
\end{document}
