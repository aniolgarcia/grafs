\documentclass[catalan, a4, 12pt]{article}
\usepackage[T1]{fontenc}
\usepackage[latin9]{inputenc}
\usepackage[catalan]{babel}
\usepackage{amsfonts}
\usepackage{xcolor}
\usepackage[a4paper]{geometry}
\usepackage[ruled, boxed]{algorithm2e}

\begin{document}

\part{Creació de grafs}
Fins ara, s'han mostrat i estudiat grafs que ja estan definits i que s'ha de fer alguna sobre ells. Ara bé, en aplicacions reals de teoria de grafs, sovint no hi ha un graf determinat, sinó que s'ha de generar. En aquest cas, segurament hi haurà alguns nodes definits (però s'en podràn afegir més) i el problema consistirà a trobar les arestes. Un exemple d'aquest cas consistiria en haver d'unir tres ciutats amb carreteres de tal manera que des d'una es pugui arribar directament a les altres dues. La primera sol·lució en que sol pensar una persona és fer un triangle en el qual les ciutats siguin els vèrtexs (l'equivalent a un graf $K_{3}$), però si es vol construïr el mínim de carreteres possibles, aquesta sol·lució no és la més eficient.


\end{document}
