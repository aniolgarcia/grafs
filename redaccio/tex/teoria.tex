%% LyX 2.1.2 created this file.  For more info, see http://www.lyx.org/.
%% Do not edit unless you really know what you are doing.
\documentclass[english]{article}
\usepackage[T1]{fontenc}
\usepackage[latin9]{inputenc}
\usepackage{amstext}
\usepackage{babel}
\begin{document}
Un graf $G=(V,E)$ es defineix com un conjunt de v�rtex (o nodes)
$V=\{v_{1},v_{2},...,v_{n}\}$ i un conjunt d'arestes $E=\{e_{1},e_{2},...,e_{m}\}$,
que uneixen dos v�rtexs $v_{i}$ i $v_{j}$, tals que $v_{i},v_{j}\text{\ensuremath{\in}}V$.
�s a dir: un graf est� format pe un conjunt de punts i un conjunt
d'arestes que uneixen alguns d'aquests punts. El nombre de v�rtexs
d'un graf queda determinat pel nombre d'elements que hi ha en el grup
$V$, per tant ens referirem a ell com a $|V|$(cardinal de $V$).
Amb les arestes passa el mateix, i tamb� utilitzarem $|E|$per determinar
el nombre d'arestes d'un graf. Definim tamb� que dos v�rtexs s�n adjacents
si est�n units per una aresta, i com a conseq��ncia, s�n incidents
a l'aresta. 

En la figura (index de la figura) es mostra un graf simple $G$ format
per:
\begin{itemize}
\item $V=\{v_{1},v_{2},v_{3},v_{4},v_{5},v_{6}\}$
\item $E=\{(v_{1},v_{2}),(v_{1},v_{4}),(v_{1},v_{5}),(v_{2},v_{5}),(v_{3},v_{4}),(v_{3},v_{5}),(v_{3},v_{6}),(v_{4},v_{6})\}$
\item (adjuntar fiura del garf anterior)
\end{itemize}
Si una aresta comen�a i acaba en el mateix v�rtex (per exemple $e_{m}=\{v_{i},v_{i}\}$)
s'anomena lla�. Tamb� pot ser que hi hagi dues arestes id�ntiques,
�s a dir, dues arestes que comencin en el v�rtex $v_{i}$ i acabin
en el v�rtex $v_{j}$. En qualsevol d'aquests dos casos anteriors,
el graf s'anomena multigraf o pseudograf. En cas contrari, el graf
ser� simple i sim�tric. Amb el que hem vist fins ara, podem dir que
$e_{1}=(v_{1},v_{2})$ �s equivalent a $e_{2}=(v_{2},v_{1})$, per�
en els grafs dirigits aix� no es compleix. En aquest tipus de graf,
les arestes nom�s permeten viatjar en un sentit. En la seg�ent imatge
es mostren els grafs esmentats anteriorment:

(Adjuntar imatge de graf amb lla�os, graf amb arestes m�ltiples i
graf dirigit)



\emph{Nota de l'autor: a partir d'ara, i si no s'indica el contrari,
quan es parli de grafs, s'exclour�n els multigrafs i grafs dirigits.}



El nombre d'arestes que s�n incidents a un v�rtex $v$ (contant els
lla�os com a dues arestes) determinen el grau de $v$, que es representa
amb $d(v)$. La successi� de graus d'un graf ser� la successi� que
s'obt� al ordenar de manera creixent els graus dels seus v�rtex. El
grau m�nim d'un graf $G$ queda determinat de la seg�ent manera: $\delta(G)=min\{d(g):v\text{\ensuremath{\in}}V(G)\}$.
De manera similar, el grau m�xim de $G$, $\text{\ensuremath{\Delta}}(G)=max\{d(v):v\text{\ensuremath{\in}}V\}$.
Amb tots aquests conceptes ja podem veure el teorema d'Euler, un dels
primers teoremes en teoria de grafs i un dels m�s importants.


\subparagraph{Teorema 1 (Euler)}
\begin{quotation}
``En tot graf, la suma dels graus dels v�rtex �s igual al doble del
nombre d'arestes.''

\[
\sum_{vV}
\]

\end{quotation}

\end{document}
