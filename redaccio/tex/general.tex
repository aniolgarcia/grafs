%% LyX 2.1.4 created this file.  For more info, see http://www.lyx.org/.
%% Do not edit unless you really know what you are doing.
\documentclass[english]{article}
\usepackage[T1]{fontenc}
\usepackage[latin9]{inputenc}
\usepackage{babel}
\begin{document}




Aquest primer apartat consisteix en la part més teòrica del treball.
Explicarè breument l'història d'aquesta branca de la matemàtica i
posteriorment ens endinsarem en la teoria de grafs com a tal. La teoria
de grafs pot ser bastant abstracta i a vegades complicada d'entendre,
però estarà acompanyada de demostracions i exemples propis que pretenen
facilitar el seguiment del treball. 


\section{Història del grafs}


\subsection{Els primers passos}

Tot sovint, les noves branques de la matemàtica sorgeixen de solucions
a problemes. Problemes que no poden ser resolts ni demostrats amb
el que coneixem, que forcen a desenvolupar nous mètodes i teories.
La teoria de grafs no n'és una excepció i tot seguit presentaré els
problemes determinants per a la creació d'aquesta branca.


\subsubsection*{Euler i els ponts de Königsberg}

La teoria de grafs neix a partir de la solució de Leonhard Euler d'un
problema curiós. Aquest matemàtic va reoldre el problema dels ponts
de la ciutat de Königsberg (l'actual Kaliningrad, Rusia), que diu
així :
\begin{quote}
\emph{``El riu Pregel divideix Königsberg en quatre parts separades,
i connectades per set ponts. És possible caminar per la ciutat passant
per tots els ponts tan sols una vegada?''}
\end{quote}
Cap dels ciutadans de Königsberg ho havia aconseguit, i ja sabien
que no era possible, però mai ningú ho havia demostrat fins que Euler
ho va fer. La demostració de que això no era possible queda recollida
en el \emph{``Solutio problmatis ad geometriam situs pertinentis}''
publicat el 1736, i l'article també va ser inclòs en el volum 8 de
\emph{``Commentarii Academiae Scientiarum Imperialis Petropolitanae''
}publicat el 1741.\emph{ }Per fer aquesta demostració, Euler va haver
de representar el problema com un mapa topològic, posant les masses
de terra com a punts i els ponts com a segments que unien aquests
punts, creant d'aquesta manera el primer graf de l'història. Aquest
resulat es considera el primer en toeria de grafs, ja que conté un
important teorema d'aquesta branca. A més d'iniciar la teoria de grafs,
amb aquest resultat també comença l'estudi dels grafs planars, introdueix
el concepte de característica d'Euler de l'espai i el teorema de poliedres
d'Euler (teorema que després va utilitzar per demostrar que no hi
havien mes sòlids regulars que els sòlids platònics). Amb tot això,
Euler posa les bases no tan sols de l'estudi dels grafs, sinó també
de la topología, una altra branca que també serà tractada en aquest
treball. 


\subsubsection*{Vandermonde i el tour del cavall}

A partir de l'article d'Euler, diversos matemàtics van començar a
interessar-se pel camp de la topologia (o geometria de la posició,
com li deien en aquell moment). Concretament hi ha un personatge important:
Alexandre-Théophile Vandermonde. Vandermore va treballar i estudiar
el problema dels cavalls, que pregunta quins moviments hem de fer
per tal que un cavall passi per totes les caselles del tauler d'escacs,
problem a que també va tractaar Euler. Els estudis que va fer sobre
aquest problema van ser publicats el 1771 en el ``\emph{Remarques
sur des problèmes de situation''}, i per la relativa proximitat als
treballs d'Euler, encara no parlava de grafs, tot i que ara el problema
es resol mitjançant aquests. Aquest treball inicia l'estudi de la
teoria de nusos, una altra branca de la topologia. 


\subsection{Les primeres descobertes i aplicacions}

Durant el segle XIX


\subsubsection*{Francis Guthrie}

El 1852 aqust matemàtic britànic es planteja el següent problema mentres
intenta pintar un mapa del regne unit:
\begin{quote}
\emph{``És possible pintar qualsevol mapa de països de tal manera
que un país tingui un color diferent al de tots els seus veïns, utilitzant
tan sols quatre colors?''}
\end{quote}
D'aquest problema en surt el teorema de que qualsevol mapa pot ser
pintat únicament amb quatre colors diferents, de tal manra que dues
regions adjacents no tinguin colors iguals. Aquest problema que pot
semblar tan trivial no va ser demostrat fins l'any 1976. Va passar
per mans de personatges com De Morgan, Hamilton, Cayley, Kempe (que
va fer una demostració publicada el 1879), Heawood (que va demostrar
que la demostració de Kempe no era correcta)... Finalment el 1976
Appel i Hanken van demostrar a través d'un programa d'ordinador que
tot mapa es podia pintar només amb quatre colors. Pel fet de basar
la demostració en un programa d'ordinador, molta gent no va acceptar
la demstració. Així doncs, aquest problema no va ser solucionat de
manera formal 1996 quan, recorrent a la teoria de grafs ja desenvolupada,
Neil Robertson, Daniel P. anders, Paul Seymour i Robin Thomas van
publicar un demostració. En els treballs d'Appel i Hanken es van definir
alguns dels conceptes i fonaments de l'actual teoria de grafs. 


\subsubsection*{Arthur Cayley}

Arthur Cayley, matemàtic que treballava en la teoria de grups, tpologia
i combiantoria, també va aportar una gran quantitat de coneixement
a la branca. Va treballar amb grafs de tipus arbre i va desenvolupar,
la fòrmula $n^{n-2}$, que determina les nombre d'arbres expansius
que té un graf complet de $n$ vèrtex. Una fòrmula semblant apareixia
en treballs de Carl Wilhelm Borchardt, en els quals Cayley es va basar
i va extendre, però el que actualment dóna nom a la fòrmula és el
mateix Cayley. 

També va treballar en desenvolupar una representació de l'estructura
abstracta d'unc grup, creant els grafs de Cayley i el teorema de Cayley
referent a aquests. Finalment, Cayley va contribuïr també el 1857
en la representació i enummeració dels isòmers alcans (composts químics
que comparteixen fòrmula o composició però tenen diferent estructura
molecular), representant cada compost mitjançant un graf de tipus
arbre. Tot i això, Cayley no només va ser actiu en teoria de grafs,
sinó que també va desenvolupar teoremes en àlgebra linea, topologia
i geometreia.


\subsubsection*{William Hamilton i Thomas Kirkman }

William Rowan Hamilton va plantejar un problema el 1859 que consistia
en trobar un camí que passés pels 20 vértex d'un dodecahedre una sola
vegada a través de les seves arestes. Hamilton va comercissalitzar
el joc sota el nom de ``The Icosian game'' (és important dir que
el nom de icosian no va ser degut a que utilitzés un icosahedre, sninó
que feia referència als 20 vértex del dodecahedre per on s'havia de
passar). Entorn aquest joc existeix un gran controvèrsia, ja que Euler
va plantejar un problema semblant mentre estudiava el problema dels
cavalls, i Kirkman va plantejar exactament el mateix problema que
Hamilton a la Royal Society un temps abans. 


\subsubsection*{Gustav Kirchhoff}

Gustav Kirchhoff, conegut majoritàriament en el camp de l'electrotècnia
per les seves lleis de Kirchhoff, també va fer aportacions importants
en teoria de grafs. Les seves lleis, publicades el 1874, es basen
en la teoria de grafs, però a més, va ser el primer d'utilitzar els
grafs en aplicacions industrials. Va estudiar sobretot els grafs de
tipus arbre i, amb l'investigació que va dur a terme sobre aquest
tipus de grafs, va formular el teorema de Kirchhoff, sobre del nombre
d'arbres d'expansió que es poden trobar en un graf. Aquest teorema
es considera una generalització de la fòrmula de Cayley.


\subsection{Teoria de grafs moderna}

Durant el segle XX, la teoria de grafs es va anar desenvolupant més.
Amb les bases ja establertes durant el segle XIX, els matemàtics hi
van començar a treballar i el 1936 Dénes König va escriure el primer
llibre de teoria de grafs. Frank Harary va escriure un altre llibre
el 1969, que va fer accessible la teoria de grafs a àmbits diferents
a les matemàtiques. El desenvolupament de l'informàtica i les noves
tècniques de computació van permetre treballar amb grafs a molt més
gran escala, fent possible, per exemple, la primera demostració del
teorema dels quatre colors per Appel i Hanken. 

Actualment la teoria de grafs és una part molt important de la matemàtica
discreta i està relacionada amb molts àmbits diferent, com per exemple
la topologia, la combinatòria, la teoria de grups, la geometria algebraica...
Des del seu desenvolupament s'han utilitzat els grafs per resoldre
i representar de manera visual problemes en aquests camps. Té aplicacions
en molts altres àmbits com per exemple la computació, l'informàtica,
la física, la química, l'electrònica, les telecomunicacions, la biologia,
la logística i fins i tot en l'àmbit econòmic. 


\end{document}
