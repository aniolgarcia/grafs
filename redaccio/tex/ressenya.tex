\documentclass[a4, 12pt, catalan]{report}
%\usepackage[T1]{fontenc}
\usepackage[latin1]{inputenc}
\usepackage[catalan]{babel}
\usepackage{amstext}
\usepackage{amsmath}
\usepackage{amsfonts}
\usepackage{mathtools}
%\usepackage[backend=biber, style=numeric, bibencoding=ISO-8859-1, defernumbers=true]{biblatex} %Llibreria de gestió de bibliografia
%\usepackage{cite}
%\usepackage{catalanbib}
\usepackage{chngcntr}
\counterwithout{footnote}{chapter}
\usepackage{xcolor}
\usepackage{graphics, graphicx}
\usepackage{tikz, tkz-graph}
\usepackage{pgf, pgfplots}
\usepackage{graphviz, tkz-berge}
\usepackage{pstricks, pst-node, pst-tree}
\usepackage[ruled, boxed]{algorithm2e}
\usepackage{multicol, multirow, rotating} %Paquets de taules
\usepackage[a4paper]{geometry}
\usepackage{float} %Imatges múltiples
\usepackage{verbatim}
\usepackage[style=english]{csquotes}
\usepackage{cancel}
\usepackage[position=top]{subfig}
\usepackage{rotating}
%\usepackage{url}
%\usepackage{minted}
\usepackage{etoolbox}
\usepackage{parskip}
\usepackage[bottom]{footmisc}
\usepackage[toc,page]{appendix}
\usetikzlibrary{arrows,%
                petri,%
                topaths}%
\usetikzlibrary{shapes}
\usetikzlibrary{arrows.meta}
\usetikzlibrary{positioning,automata}

\graphicspath{ {graphics/} }

%Definim un nou entorn "algorisme"
\newenvironment{algorisme}[1][htb]
  {\renewcommand{\algorithmcfname}{Algorisme}% Canviem Algorithm a Algorisme
   \begin{algorithm}[#1]%
  }{\end{algorithm}}
 
%Canviem el nom de Apèndix per Annex 
\addto{\captionscatalan}{\renewcommand*{\appendixname}{Annex}}

%Definim una orfre per subratllar amb color verd
\newcommand{\important}[1]{%
  \colorbox{green!25}{$\displaystyle#1$}}
    
%Definim que els comentaris siguin de color blau
\newcommand\mycommfont[1]{\footnotesize\ttfamily\textcolor{blue}{#1}}
\SetCommentSty{mycommfont}

%Definim un apartat de funció per al pseudocodi
\SetKwProg{Fn}{Funció}{}

%Definim el contador per les propietats (que fa un reset cada subsecció)
\newcounter{propietat}[subsection]
%Definim l'entorn per a les propietats
\newenvironment{propietat}[1][]{\refstepcounter{propietat}\par\medskip
  \noindent \textbf{Propietat~\thepropietat #1} \rmfamily}
   {\medskip}

%Redefinim el llaç de tikz (l'original és esquifit i no m'agrada massa), i sí, el nom que he definit és "looop", amb 3 "o". 
\makeatletter
\tikzset{looop/.style =  {to path={
  \pgfextra{\let\tikztotarget=\tikztostart}
  [looseness=12,min distance=10mm]
  \tikz@to@curve@path},font=\sffamily\small
  }}  
\makeatletter 

%Creem els conjunts per a les llistes d'adjacència
\tikzset{
node of list/.style = { 
             draw, 
             minimum height=6mm, 
             minimum width=6mm,
             node distance=6mm
   },
link/.style = {
     -stealth,
     shorten >=1pt
     },
array element/.style = {
    draw, fill=white,
    minimum width = 6mm,
    minimum height = 10mm
  }
}

\def\LinkedList#1{%
  \foreach \element in \list {
     \node[node of list, right = of aux, name=\element] {\element};
     \node[node of list, name=aux2, anchor=west] at ([xshift=-.4pt] \element.east) {};
     \draw[link] (aux) -- (\element);
     %\coordinate (aux) at (\element.east);
     \coordinate (aux) at (aux2);
  } 
   \fill (aux) circle(2pt);
}

%Definim el color taronja pel procés de punt de Fermat
\definecolor{ffwwqq}{rgb}{1,0.4,0}

%Afegim una categoria de Liblatex per tal d'agafar només la bibliografia citada (no funciona)
%\DeclareBibliographyCategory{citats}
%\AtEveryCitekey{\addtocategory{citats}{\thefield{entrykey}}}

%Importem els fitxers de la bibliografia
%\addbibresource{biblio2.bib}
%\addbibresource{biblio3.bib}

%Creem dues classes diferentes per a les entrades de la bibliografia
%\DeclareSourcemap{
%  \maps[datatype=bibtex]{
%    \map{
%      \perdatasource{biblio2.bib}
%      \step[fieldset=keywords, fieldvalue={, refer}, append]
%    }
%    \map{
%      \perdatasource{biblio3.bib}
%      \step[fieldset=keywords, fieldvalue={, biblio}, append]
%    }
%  }
%}

%Arreglem els nombres de la bibliografia ( que siguin continus)
%\makeatletter
%\define@key{blx@bib2}{prefixnumbers}{%
%  \def\blx@prefixnumbers{#1}%
%  \iftoggle{blx@defernumbers}
%   {}
%    {\iftoggle{blx@labelnumber}
%       {\blx@warning{%
%          Option 'prefixnumbers' requires global\MessageBreak
%          'defernumbers=true'}}
%       {}}}
%\makeatother

%Creem unentorn de BibLaTeX sense numeració d'entrades bibliogràfiques
%\defbibenvironment{nolabelbib}
%  {\list
%     {}
%     {\setlength{\leftmargin}{\bibhang}%
%      \setlength{\itemindent}{-\leftmargin}%
%      \setlength{\itemsep}{\bibitemsep}%
%      \setlength{\parsep}{\bibparsep}}}
%  {\endlist}
%  {\item}

%definim la mida de la lletra dels programes en 11 ( causa diversos warnings)
\AtBeginEnvironment{minted}{\singlespacing%
    \fontsize{11}{11}\selectfont}

%Posem el contador de capítols a -1, de manera que comenci per 0
\setcounter{chapter}{-1}

%Per posar l'interlineat a (1.5). Recomanable comentar-ho i deixar l'interlineat per defecte...
%\renewcommand{\baselinestretch}{1.5}

%Afegim l'entrada d'annexos a la llibreria de idioma
\addto\captionscatalan{%
   \renewcommand{\appendixtocname}{Annexos}%
   \renewcommand{\appendixpagename}{Annexos}%
}

\interfootnotelinepenalty=10000

\newenvironment{bottompar}{\par\vspace*{\fill}}{\clearpage}

\title{\Huge \textbf{Els grafs: xarxes, camins i connexions} \\ \LARGE De la matemàtica discreta a la realitat}
\author{\LARGE Aniol Garcia i Serrano \\ Tutor: Xavier Aguilera Colmenero\\ \small 2n Batxillerat}
\date{Curs 2016-2017 \\ 21/11/2016}

\begin{document}
\begin{center}
    \Large
    \textbf{Els grafs: xarxes, camins i connexions.}
    
    \vspace{0.1cm}
    \large
    De la matemàtica discreta a la realitat
    
    \vspace{0.3cm}
    \textbf{Aniol Garcia i Serrano \\ \small Tutor: Xavier Aguilera Colmenero}
\end{center}

\section*{Presentació}
El tema d'aquest treball, tal com diu el títol, tracta sobre els grafs: xarxes, camins i connexions. Pretén fer el pas de la matemàtica discreta a la realitat. La matemàtica discreta s'escarrega de l'estudi de conjunts, estructures i processos formats per elements que es poden comptar un a un i de manera separada. La teoria de grafs n'és una branca. Els grafs sovint formen part de l'entramat de mecanismes que fan funcionar moltes de les coses que ens envolten, que ens permeten relacionar-nos, o bé que ens faciliten el dia a dia, per exemple. Però, malgrat tot, són uns grans desconeguts. Al llarg del treball intento crear lligams entre la part més teòrica i abstracta i algunes de les aplicacions que se'n deriven. Intento endinsar-me en el coneixement per comprendre aquests mecanismes. Intento crear els mecanismes per que el coneixement esdevingui una eina pràctica i funcional. 

\section*{Estructura i Metodologia}
El cos del treball s'estructura en quatre capítols que permeten endinsar-se, de manera seqüenciada, en el coneixement del grafs i les seves aplicacions.

En el primer capítol es fa una introducció a la teoria de grafs. Es comença fent un recorregut per la història dels grafs, des dels seus orígens, amb el planteig dels primers problemes, fins la teoria de grafs moderna. Tot seguit s'expliquen els conceptes bàsics: és la part més teòrica en la qual es defineix el graf i es fa referència a la seva estructura i els seus components. S'explica també què són els isomorfismes i, finalment, es fa una descripció dels diferents tipus de grafs. Cal remarcar que, en aquest apartat, s'han exposat les propietats fonamentals i s'han inclòs demostracions  sempre que ha estat possible. Majoritàriament, aquestes demostracions no han estat fruit d'una recerca a la xarxa sinó d'una recerca personal. Les persones i entitats a les qui adreço part dels meus agraïments a l'inici d'aquest treball hi han tingut molt a veure.

En el segon capítol es fa un pas més enllà: es descriu què són els grafs ponderats i els dirigits, els tipus de camins i les maneres diferents de gestionar tota aquesta informació afegida. Finalment s'estudien diferents algorismes segons la seva finalitat. S'analitza el funcionament, les propietats fonamentals i les aplicacions més comuns de cadascun d'aquests procediments i s'especifica el pseudocodi. En l'Annex B adjunto els programes, amb llenguatge Python, que permeten resoldre els algorismes, juntament amb exemples d'entrada i de sortida. La implementació d'aquests algorismes és, també, una aportació personal.

El tercer capítol és un incís al disseny de grafs. Es fa referència al punt de Fermat i a l'arbre de Steiner per la seva rellevància. Se'n mostra un bonic exemple mitjançant les bombolles de sabó.

Finalment, en el darrer capítol, s'exposen algunes aplicacions pràctiques. Una d'elles és el recorregut per la xarxa de metro de Barcelona. Aporto al treball un algorisme que, a partir d'una estació de sortida i una altra d'arribada, ofereix, com a resposta, el trajecte que permet fer el recorregut amb el mínim temps possible. Aquesta resposta inclou les estacions per on es passarà, els transbordaments que caldrà fer i el temps total que es trigarà per anar d'un punt a l'altre. En l'Annex B adjunto la codificació d'aquest algorisme així com exemples d'entrades i sortides. 

Les meves conclusions i valoracions esdevenen el punt i final d'aquest treball. 

Les fonts d'informació utilitzades han estat molt diverses. En la bibliografia he detallat els documents i les pagines Webs consultades. Vull remarcar, però, que el gruix més important de la informació prové d'altres fonts que per a mi han estat més significatives pel fet que han estat més properes i, sobretot, vivencials: el guiatge de la UB per part del Dr. Antoni Benseny i, sobretot, el seguiment i l'acompanyament d'en Robert Salla han estat claus; la participació en el Math Summer Camp\footnote{Activitat organitzada per Fundació Privada Cellex, UPC-FME i CFIS} i en el programa Bojos per les matemàtiques\footnote{Programa organitzat per FEEMCAT, SCM i Fundació Catalunya La Pedrera} també em va aportar moltíssima informació; les xerrades i consells per part dels investigadors de l'IRI m'han obert moltes portes; les converses amb l'Anton Aubanell, imprescindibles. 

\section*{Cos del treball}
Ni faba de què posar aquí.
\section*{Conclusions}
Durant aquest treball hem fet un camí juntament amb la teoria de grafs: l'hem vist néixer del pensament de Leonhard Euler; l'hem vist créixer acompanyada d'alguns dels més grans matemàtics de l'història; n'hem vist la seva manera d'entendre i modelitzar el món; l'hem acompanyada fins a la seva maduresa on, juntament amb més matemàtics i teòrics de les ciències de computació, ha col·laborat amb una infinitat d'àrees que en un principi semblaven distants a ella; i finalment ens ha ajudat a resoldre problemes quotidians.

Un cop acabat el treball, puc afirmar que he assolit els objectius proposats. D'una banda, m'he endinsat en el coneixement de la teoria de grafs començant per la seva història, que m'ha permès conèixer l'evolució. He definit diferents tipus de grafs i les seves propietats. Ha calgut incloure demostracions i me n'he adonat que són una part molt important de les matemàtiques. He demostrat, a tall d'exemple, la relació que s'estableix entre el nombre de nodes i d'arestes d'un graf lineal respecte el graf original, o bé l'expressió matemàtica que defineix el nombre d'arestes d'un graf xarxa. En aquest punt, com en tants d'altres, ha estat important compartir coneixements amb persones enteses en la matèria, la qual cosa m'ha permès d'anar avançant amb confiança. 

D'altra banda, també he aconseguit estudiar amb certa profunditat i formalitat alguns dels algorismes més usats en aquest camp. N'he estudiat el temps d'execució, les propietats i les possibles aplicacions pràctiques i, de tots ells, n'he fet una implementació en Python, recollida a l'annex B. A més, també he implementat altres algorismes que no es tracten durant el treball, però que estàn estretament relacionats amb conceptes d'aquest, com els algorismes per trobar camins Eulerians i Hamiltonians en un graf. Podríem dir que he ampliat es objectius incials al generar nous algorismes i aplicar-los a problemes reals. 
El capítol 4 és la concreció de tots aquests objectius mitjançant una cas pràctic: a partir del coneixement del graf es planteja una situació que cal resoldre. Es crea un algorisme, s'aporten unes dades a partir d'un treball de camp i, finalment, s'obté un resultat que ens aporta la informació desitjada.

Amb els objectius acomplerts, la hipòtesi queda verificada. La teoria de grafs ens proporciona eines per modelitzar estructures i processos i ens permet crear aplicacions pràctiques mitjançant procediments algorísmics. Hem vist que podem modelitzar una gran quantitat d'estructures: des de xarxes de telecomunicacions, fins a representar com a graf les dependències de cadascun dels apartats d'un treball, passant per altres situacions com per exemple modelitzar la xarxa de metro. Un cop els models estan definits, ens trobem amb tot un assortiment d'algorismes informàtics que permeten extreure dades de models teòrics i transformar-los en informació útil, resultats. Mitjançant els algorismes hem aconseguit solucionar el problema de la coloració de grafs, ordenar aquest treball de manera òptima i poder conèixer el recorregut i temps de viatge entre dues estacions qualssevol del metro de Barcelona, entre d'altres.

Com a conclusions, a més d'assolir els objectius i verificar la hipòtesi inicial, m'agradaria afegir que:
\begin{itemize} 
  \item Els procediments algorísmics tenen un component matemàtic molt important. Alguns es basen en conceptes matemàtics més generals, com per exemple l'algorisme de Dijkstra amb la desigualtat triangular; en canvi d'altres, com el de Prim, es basen en propietats singulars dels grafs.
  \item Els grafs, no només formen part de la realitat que ens envolta si no que, a més, ens faciliten moltes gestions i ens proporcionen comoditats de les quals ens seria difícil prescindir: les xarxes socials, Internet, el subministrament elèctric, navegadors GPS...  
  \item Implementar algorismes, a més de la part matemàtica, en ocasions requereix un treball de camp i/o una recollida de dades, com ha estat el cas de l'algorisme de la Xarxa de metro de Barcelona: per tal de treballar amb valors precisos ha estat necessari mesurar els temps reals de viatge entre les estacions de totes les línies.

\end{itemize}


\end{document}