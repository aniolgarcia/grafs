%% LyX 2.1.2 created this file.  For more info, see http://www.lyx.org/.
%% Do not edit unless you really know what you are doing.
\documentclass[english]{article}
\usepackage[LGR,T1]{fontenc}
\usepackage[latin9]{inputenc}
\usepackage{amstext}

\makeatletter

%%%%%%%%%%%%%%%%%%%%%%%%%%%%%% LyX specific LaTeX commands.
\DeclareRobustCommand{\greektext}{%
  \fontencoding{LGR}\selectfont\def\encodingdefault{LGR}}
\DeclareRobustCommand{\textgreek}[1]{\leavevmode{\greektext #1}}
\DeclareFontEncoding{LGR}{}{}
\DeclareTextSymbol{\~}{LGR}{126}

\makeatother

\usepackage{babel}
\begin{document}

\part*{Arbres}

Els arbres es defineixen com a grafs connexos sense cicles, de tal
manera que hi ha un �nic cam� entre dos nodes qualsevols. Per aix�
es diu que si $G$ �s un arbre, tan sols existeix un cam� elemental
entre $u,v$ \ensuremath{\in} $V(G)$. Si a un arbre se li afegeix
una aresta, genera cicles, i per tant deixa de ser un arbre. De la
mateixa manera, si es treu una aresta, el graf deixa de ser connex. 


\subsection*{Proposicions}


\paragraph{Proposici� 1: Si $T$ �s un arbre amb $n\geq2$, $\delta=1$ i hi
ha com a m�nim 2 nodes de grau 1.}

Demostraci� 1: Tots els nodes d'un arbre tenen tan sols un ``pare'',
per tant, tots els nodes que no tinguin ``fills'' ser�n de grau
1. Si aix� no es compl�s, voldria dir que hi ha cicles. Com que hi
haur� un noment que l'arbre deixi de cr�ixer, els �ltims nodes ser�n
de grau 1. De la mateixa manera, com que el graf t� com a m�nim 2
nodes i no pot decr�ixer, haur� de tenir com a m�nim dues branques,
i pel que hem dit abans, tamb� dues fulles. 


\paragraph*{Proposici� 2: El nombre de fulles de $T$ �s m�s gran o igual que
$\text{\textgreek{D}}$}

Demostraci� 2: Del node amb grau $\text{\textgreek{D}}$ en sortir�n
$\text{\textgreek{D}}$ nodes. Com que el graf no pot decr�ixer ni
cr�ixer infinitament, cadascun d'aquests nodes tindr� o b� una fulla
o b� una branca, que acabar� en una fulla.


\paragraph{Proposici� 3: Tot graf connex cont� un arbre amb el mateix nombre
de nodes}

Demostraci� 3: Tot el que priva a un graf de ser un arbre s�n els
cicles. Si es treu una aresta d'un cicle, el graf segueix siguent
conex. Si anem trient arestes dels cicles del graf sense fer-lo disconex.
\end{document}
